\defverbatim[colored]\inputfail{%
\begin{lstlisting}[language=Html]
<input type="text" name="initdate"
	onchange="validateDate(this);" />


<input type="text" name="enddate"
	onchange="validateDate(this);" />
\end{lstlisting}}%

\defverbatim[colored]\inputok{%
\begin{lstlisting}[language=Html]
<input type="text" name="initdate"
	class="date" />


<input type="text" name="enddate"
	class="date" />
\end{lstlisting}}%

\defverbatim[colored]\inputjs{%
\begin{lstlisting}[language=Java]
window.onload = function(){
    inputs = document.getElementsByTagName('input');
    for(var i=0,l=inputs.length;i<l;i++){ 
        if(inputs[i].className
			&& inputs[i].className=='date'){ 
            input.onchange = function(){ 
                validateDate(this);
            }
        }
    }
}; 
function validateDate(element) { }
\end{lstlisting}}%

\section{Javascript Não-Obstrusivo}
\begin{frame}{Javascript Não-Obstrusivo}
\inputfail
\begin{itemize}
	\pause \item E se não estiver com javascript habilitado?
	\pause \item E se o nome da função javascript mudar?
	\pause \item E se for necessário adicionar um novo parâmetro?
\end{itemize}
\end{frame}

\begin{frame}{Javascript Não-Obstrusivo}
\inputok
\begin{itemize}
	\pause \item E se o HTML apenas avisar "este campo contém uma data" ?
\end{itemize}
\end{frame}

\begin{frame}{Javascript Não-Obstrusivo}
\inputjs
\end{frame}

\begin{frame}{Javascript Não-Obstrusivo}
\begin{itemize}
	\item Qualquer input com a classe "date" será validado
	\item Fácil manutenção com códigos mais limpos
	\item Separação das camadas (apresentação, conteúdo e interação)
	\item Com javascript desabilitado, não há validação, mas também não há erros
\end{itemize}
\begin{description}
	\item[HTML] Responsável somente pelo conteúdo
	\item[CSS] Responsável pela apresentação
	\item[Javascript] Responsável pela interação com o usuário
\end{description}
\end{frame}

\begin{frame}{Javascript Não-Obstrusivo}
\begin{itemize}
	\item Melhores práticas para resolver os típicos problemas cross-browsers como progressive enhancement/graceful degradation
	\item Não suponha que o JavaScript estará habilitado, otimize seu código para não ficar dependente dele;
	\item Não suponha que os browsers interpretarão corretamente determinados métodos e propriedades, teste individualmente em cada browser antes de publicar;
	\item Não suponha que o código HTML estará correto, verifique-o e não faça nada até que ele esteja devidamente estruturado;
	\item Mantenha a funcionalidade independente do dispositivo do qual seu site é acessado;
	\item Suponha que outros scripts tentarão interferir com o seu e mantenha o seu script o mais seguro possível.
\end{itemize}
\end{frame}
