\section{Começando do Começo}
\begin{frame}{Começando do Começo}
\begin{block}{window.onload vs \$(document).ready()}
\begin{itemize}
	\item \textit{window.onload} é executando quando alguns elementos ainda não foram criados
	\item \textit{window.onload} somente pode ser usado uma vez
	\item \textit{\$(document).ready()} é executando quando toda página foi carregada
	\item \textit{\$(document).ready()} vai agregando funções
\end{itemize}
\end{block}
\begin{block}{Logo...}
	\$(document).ready(function() \{ \\
		// código \\
	\}); \\
	ou \\
	\$(function() \{
		// código \\
	\});
\end{block}
\end{frame}

\begin{frame}{Começando do Começo}
\begin{itemize}
	\item Tudo começa com um seletor...
	\pause \item Que se torna um Array de elementos que casam com este seletor
	\pause \item Cada método é executado para todos os elementos selecionados
\end{itemize}
\end{frame}
