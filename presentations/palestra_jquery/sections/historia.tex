\section{Informações}
\begin{frame}{jQuery}
	\begin{center}
		{\Huge jQuery} \\
		{\tiny Write Less, Do More}
	\end{center}
\end{frame}

\begin{frame}{Informações}
\begin{itemize}
	\item Jquery é um framework criado por john Resig que facilita a interação entre javascript e html
	\item Tem foco na simplicidade. Por que desenvolver longos e complexos códigos para simples efeitos?
	\item Primeira versão, 1.0a, foi lançada em junho de 2006
	\item Primeira versão estável, 1.0, lançada em agosto de 2006
	\item Última versão é a 1.3.2, lançada em fevereiro de 2009
	\item Versão 1.4 programada para até o fim de 2009
	\item Todo framework tem 19KB minificado e com gzip
	\item Compatível com IE6+, FF2.0+, Safari 3.0+, Opera 9.0+, Chrome
\end{itemize}
\end{frame}

\section{Áreas}
\begin{frame}{Áreas}
\begin{description}
	\item[Core] Plugins, interoperabilidade, necessário para funcionar
	\item[Selectors] Seletores de elementos do DOM
	\item[Atributos] Manipulação de atributos do DOM
	\item[Traversing] Percorrer os elementos DOM
	\item[Manipulation] Manipulação dos elementos DOM
	\item[CSS] Manipulação de propriedades CSS dos elementos do DOM
	\item[Events] Eventos do DOM
	\item[Effects] Efeitos
	\item[Ajax] Requisições síncronas e assíncronas
	\item[Utilities] Funções para utilização geral, que facilitam o desenvolvimento
	\item[UI] Integração para User Interface (pacote de ícones, botões, estilos, elementos, ...)
\end{description}
\end{frame}
