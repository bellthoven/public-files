\section{Templates}
\begin{frame}{Templates}
	\begin{block}{Funcionalidade}
		Permite que, ao abrir um novo arquivo, o arquivo tenha um template padrão
		\begin{itemize}
			\item Acelerando o desenvolvimento
			\item Certificando-se de que não será esquecido de nada
			\item Menos erros de digitação
			\item Evita o raciocínio e a memorização sobre coisas desnecessárias
			\item Padroniza documentos
		\end{itemize}
	\end{block}
	\begin{block}{Utilidade}
		\begin{itemize}
			\item Criar template para uma extensão de arquivo
			\item Criar template para um arquivo que contenha uma certa palavra
			\item Criar template para um arquivo que esteja dentro de um certo diretório
		\end{itemize}
	\end{block}
\end{frame}

\subsection{Exemplos}
\begin{frame}{Exemplos}
	\begin{enumerate}
		\item Criar o arquivo bash.template com o template desejado
		\item Colocar no .vimrc o código para carregar o template para todos os arquivos com extensão .sh
		\item Sentir a magia
	\end{enumerate}
	\begin{block}{bash.template}
		\#!/bin/bash
	\end{block}
	\begin{block}{.vimrc}
		autocmd BufNewFile *.sh 0r bash.template
	\end{block}
	\begin{block}{Shell}
		\$ vim teste.sh
	\end{block}
\end{frame}
