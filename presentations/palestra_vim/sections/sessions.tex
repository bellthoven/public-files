\section{Sessions}
\begin{frame}{Sessions}
	Imagine que você está em casa programando. Chega sua namorada e diz: "amor, desliga esse computador e vamos pro quarto". O que fazer?\\
	\pause \large Respostas:
	\begin{enumerate}
		\item Desliga o computador pressionando o botão pra ir mais rápido, mais tarde é só reabrir os arquivos e lembrar de onde parou
		\pause \item Diz pra ela que agora não pode, pois tem 10 arquivos abertos, 1 diff e está no meio de um algoritmo complexo
		\pause \item Finge que não escutou nada
		\pause \item Desliga o monitor e reza pra que ninguém mais mexa no computador
		\pause \item Salva a sessão e continua da onde parou quando quiser
	\end{enumerate}
\end{frame}
\begin{frame}{Sessions}
	Sempre que se abre o vim, se inicia uma nova sessão. E nela são gravados:
	\begin{itemize}
		\item Hitórico de comandos
		\item Históricos de \textit{undos}
		\item Arquivos abertos em buffers
		\item Arquivos abertos em abas
		\item Mapeamento de teclas
		\item Abreviaturas\ldots
	\end{itemize}
	\begin{block}{Como usar?}
		:mksession sessions/algoritmo\_X.vim \\
		\$ vim -S sessions/algoritmo\_X.vim
	\end{block}
\end{frame}
