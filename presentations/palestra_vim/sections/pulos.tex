\section{Pulos}
\begin{frame}{Pulos}
	\begin{description}
		\item[gg] Primeira linha do arquivo
		\pause \item[G] Última do arquivo 
		\pause \item[\textasciicircum] Primeiro caracter não nulo
		\pause \item[\$] Último caracter não nulo
		\pause \item[b] Primeiro caracter da palavra acima do cursor
		\pause \item[e] Última caracter da palavra acima do cursor
		\pause \item[fx] Primeira incidência depois do cursor de \textit{x} na linha
		\pause \item[Fx] Primeira incidência anterior ao cursor de \textit{x} na linha
	\end{description}
\end{frame}
\begin{frame}{Pulos por Marcas}
\begin{block}{Quando usar?}
	\begin{itemize}
		\item Quando se é difícil encontrar algum trecho específico do arquivo
		\item Quando se precisa apenas um trecho de vários arquivos pra se escrever um outro
		\item Quando o arquivo é muito longo e precisa ser scrollado
		\item \textless{}Insira aqui a sua utilidade\textgreater
	\end{itemize}
\end{block}
\end{frame}
\begin{frame}{Pulos por Marcas}
\begin{block}{Como usar?}
	\begin{description}
		\item[ma] Marca a letra \textit{a} neste ponto.
		\begin{itemize}
			\item Marca a linha cujo cursor está em cima.
			\item Pode-se utilizar qualquer uma das 26 letras.
			\item São 26 letras por arquivo aberto.
			\item Devem ser em minúsculas.
		\end{itemize}
		\item[mA] Marca a letra \textit{A} neste ponto.
		\begin{itemize}
			\item Marca a linha cujo cursor está em cima.
			\item Pode-se utiilziar qualquer uma das 26 letras.
			\item São 26 letras por sessão.
			\item Devem ser em minúsculas.
			\item São visíveis de qualquer arquivo
		\end{itemize}
		\item['a] Pula para a marca \textit{a} (mesmo arquivo)
		\item['A] Pula para a marca \textit{A} (mesma sessão)
	\end{description}
\end{block}
\end{frame}
